\chapter{Conclusion}\label{Conclusion}

In this work a $\Delta\Sigma$ modulator was implemented in transistor-level based on the behavioural model derived in the feasibility study. The modulator was designed in a 180nm technology at an operating voltage of 1.8V. The final design is a third-order, fully differential, $\Delta\Sigma$ modulator with an oversampling ratio of 170. Through nominal and PVT simulations the robustness of the modulator was validated. It met the performance requirements with approximately reaching a resolution of 16 bit with a SINAD greater than 93 dB.

The OTAs were designed using folded cascode topology, which ended up with stratifying the specifications given in the feasibility study. One thing to notice is that the power consumption of OTA 1 seems to have a consumption greater than the rest of the OTAs. The use of transistors in moderate region have indeed proven to avoid problems involving limiting the output swing. Furthermore the use of $g_m/I_D$ methodology has made it easier and faster to find the suitable dimensions for the transistors by characterizing the performance of NMOS and PMOS transistors in all regions of operation. A wide swing current mirror was used to bias the transistors, as the topology has the perks of not limiting the signal swing as much as the other conventional current mirrors. To control the common mode output, a switched capacitor CMFB circuit was implemented. It was seen that the stability did not degrade much.  

The quantizer was implemented as a latched comparator, since its resolution is 1-bit. Because of the low resolution, a topology that focus more on high-speed operation than accuracy was chosen. The hysteresis was the main parameter of concern, and it was found to be as small as 586$\mu V/V$. A two-phase non-overlapping clock generator circuits with delayed phases was implemented with the intention of operating the comparator and the TG switches in a synchronized manner. The on-resistance of the TG switches was also a concern. Through an exhaustive iteration of tuning the dimensions of the transistors in the TG switches, a set of values of the on-resistance were found. Although these values diverged some from the ones given by the specification, there were no noticeable degradation in the performance of the modulator. 

\section{Future work}
This section describe possible steps that can be done to further improve, analyze and expand the $\Delta\Sigma$ ADC in both system and circuit level. 

\subsection{System level}
This thesis has provided useful insight about the flow of designing and implementing a $\Delta\Sigma$ ADC. It could be especially interesting to implement and compare a second order $\Delta\Sigma$ modulator and compare it with this work, since it has advantages such as relaxing stability and reduced area usage. The methodology presented in the feasibility study and the thesis could easily be used as a guidance to implement the modulator. 

The next step in the design process of the ADC could be to integrate the modulator with other blocks in the receive path to demonstrate an overall system. One of these blocks required to complete the ADC is a decimation filter, which is a challenge in itself. The decimation filter performs two important tasks. One being filtering out frequency components above $f_b$ from the output signal from the quantizer, the other is down-sampling the data from the oversampling frequency to the Nyquist rate. Another thing to look at is a way of generating the three bias currents used in the OTAs. 

Monte Carlo simulations of the modulator could be useful for deriving a model that includes the statistical variation of the parameters. It would both help in finding the statistical specifications of the integrators and verify that the mismatch effects do not affect the modulator performance. 

\subsection{Circuit level}

It was observed that the power consumption of OTA 1 was a little high. A way of reducing it, an adaptive bias circuit with the folded cascode architecture can be used which improves the power efficiency while achieving high slew rate. It will operate in class AB. 

Monte Carlo simulations should be done for the OTAs, as the effects of mismatch on these important analog blocks are of interest.

Finally a layout should be realized of the modulator to verify that the parasitic capacitors do not affect its response. It could also be interesting to fabricate and measure a test-chip based on the modulator core.
